\Section{related}{Related Work}
% -- Applications
Many NLP applications make use of a knowledge base of facts.
These include semantic parsing
	\cite{key:1996zelle-semantics,key:2005zettlemoyer-semantics,key:2005kate-semantics,key:2007zettlemoyer-semantics}
  question answering \cite{key:2001voorhees-trec},
  information extraction
  \cite{key:2011hoffman-kbp,key:2012surdeanu-mimlre}, and recognizing textual
  entailment
  \cite{key:2010-schoenmackers-horn,key:2011berant-entailment}.

% -- OpenIE
A large body of work has been devoted to creating
  such knowledge bases.
In particular, OpenIE systems such as
  \sys{TextRunner} \cite{key:2007yates-textrunner},
  \sy{ReVerb} \cite{key:2011fader-reverb},
  \sys{Ollie} \cite{key:2012mausam-ollie},
  and \sys{NELL} \cite{key:2010carlson-nell} have tackled the task of compiling
  an open-domain knowledge base.
% -- ConceptNet
Similarly, the MIT Media Lab's \sys{ConceptNet} project
  \cite{key:2004liu-conceptnet}
  has been working on creating a large database of common sense
  facts.

% -- Extending OpenIE
% (people extending OpenIE)
There have been a number of systems aimed at automatically
  extending these databases.
That is, given an existing database, they propose new relations to be added.
\newcite{key:2006snow-wordnet} present an approach to enriching the \sys{WordNet}
  taxonomy;
  \newcite{key:2011tandon-conceptnet} extend \sys{ConceptNet} with new facts;
  \newcite{key:2010soderland-adapting} use \sys{ReVerb} extractions to enrich
  a domain-specific ontology.
% (how we differ)
We differ from these approaches in that we aim to provide an exhaustive
  completion of the database; we would like to respond
  to a query with either membership or lack of membership, rather than
  extending the set of elements which are members.

\newcite{key:2012yao-schemas} and \newcite{key:2013riedel-schemas} present
  a similar task of predicting novel relations between Freebase entities by
  appealing to a large collection of \sys{OpenIE} extractions.
Our work focuses on arguments which are not necessarily named entities,
  at the expense of leveraging less entity-specific information.

% -- GOFAI
% (intro)
Work in classical artificial intelligence has tackled the related task
  of loosening the closed world assumption and monotonicity of logical
  reasoning, allowing for modeling of unseen propositions.
% (default logic)
\newcite{key:1980reiter-logic} presents an approach to leveraging default
  propositions in the absence of contradictory evidence;
  \newcite{key:1980mccarthy-circumscription} defines a means of overriding
  the truth of a proposition in abnormal cases.
% (probabilisitc semantics)
Perhaps most similar to this work is \newcite{key:1989pearl-probabilistic}, who
  proposes approaching non-monotonicity in a probabilistic framework,
  and in particular presents a framework for making inferences which are not
  strictly entailed but can be reasonably assumed.
% (how are we different)
Unlike these works, our approach places a greater emphasis on working with
  large corpora of open-domain predicates.
