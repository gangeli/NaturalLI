%%%%%%%%%%%%%%%%%%%%%%%%%%%%%%%%%%%%%%%%%%%%%%%%%%%%%%%%%%%%%%%%%%%%%%%%%%%%%%%
% NATURAL LOGIC
%%%%%%%%%%%%%%%%%%%%%%%%%%%%%%%%%%%%%%%%%%%%%%%%%%%%%%%%%%%%%%%%%%%%%%%%%%%%%%%

%%%%%%%%%%%%%%%%%%% 
% NATLOG AS SYLLOGISMS
%%%%%%%%%%%%%%%%%%%
%\begin{frame}{\scalebox{0.95}{\uline{\textbf{Natural Logic}} Inference for Common Sense Reasoning}}
%\begin{center}
%  \teaserBlindInference
%\end{center}
%\end{frame}
%
%\begin{frame}[noframenumbering]{\scalebox{0.95}{\uline{\textbf{Natural Logic}} Inference for Common Sense Reasoning}}
%\begin{center}
%  \teaserInference
%\end{center}
%\end{frame}

\begin{frame}{Natural Logic as Syllogisms}
\begin{center}
  \hh{s/Natural Logic/Syllogistic Reasoning/g} \\
  \vspace{0.25cm}
  \begin{tabular}{lp{4cm}}
    &Some cat ate a mouse \\
    & \darkgray{\textit{(all mice are rodents)}} \\
    $\therefore$& \true{Some cat ate a \textbf{rodent}} \\
  \end{tabular}
\end{center}
\vspace{0.25cm}
\pause

\hh{Cognitively easy inferences are easy:} \\
\begin{itemize}
  \item 
    \begin{tabular}{lp{4cm}}
      &Most cats eat mice \\
      $\therefore$& \true{Most cats eat \textbf{rodents}} \\
    \end{tabular}
  \pause
\item ``\textit{All students who know a foreign language learned it at university.}''
  \pause
  \item[] $\therefore$ \true{``They learned it at school.''}
\end{itemize}
\vspace{0.25cm}
\pause

\hh{Facts are text; inference is lexical mutation}
\end{frame}

%%%%%%%%%%%%%%%%%%% 
% POLARITY
%%%%%%%%%%%%%%%%%%%
\def\catFelineVenn{
  \begin{tikzpicture}
    \def\vennA{(-0.1,-0.1) circle (0.2)}
    \def\vennB{(-0.0,-0.0) circle (0.5)}

    \draw \vennB node [below] {};
    \draw \vennA node [above] {};
    \begin{scope}
      \fill[fill=light] \vennB;
    \end{scope}
    \begin{scope}
      \fill[fill=dark] \vennA;
    \end{scope}
    
    \frameVenn
    \draw (0, -1.3) node {\w{cat} $\subseteq$ \w{feline}};
  \end{tikzpicture}
}

\def\felineAnimalVenn{
  \begin{tikzpicture}
    \def\vennA{(-0.1,-0.1) circle (0.2)}
    \def\vennB{(-0.0,-0.0) circle (0.5)}

    \draw \vennB node [below] {};
    \draw \vennA node [above] {};
    \begin{scope}
      \fill[fill=dark] \vennB;
    \end{scope}
    \begin{scope}
      \fill[fill=light] \vennA;
    \end{scope}
    
    \frameVenn
    \draw (0, -1.3) node {\w{feline} $\subseteq$ \w{animal}};
  \end{tikzpicture}
}

\def\header{
  \hh{Hypernymy is a \textit{bounded distributive lattice}.}
  \begin{center}
    \catFelineVenn \felineAnimalVenn 
    \hspace{0.5cm}
    \raisebox{1cm}[0pt][0pt]{
      \includegraphics[height=1.0cm]{../../img/rArrow.jpg}
    }
    \hspace{0.5cm}
    \scalebox{0.75}{\lattice}
  \end{center}
}
\def\blurb{
  \hh{\textit{Polarity} is the direction a lexical item can move along the lattice.} \\
}
\def\title{Natural Logic and Polarity}

\begin{frame}{\title}
  \header
  \pause
  \blurb
  \begin{center}
    \monoHeader
      \node[black]{animal};
      \node[black]{feline};
      \node[punktchain,color=blue]{cat};
      \node[black]{house cat};
    \end{tikzpicture}
  \end{center}
\end{frame}

\begin{frame}[noframenumbering]{\title}
  \header
  \blurb
  \begin{center}
    \monoUp{house cat}{cat}{feline}{animal}
  \end{center}
\end{frame}

\begin{frame}[noframenumbering]{\title}
  \header
  \blurb
  \begin{center}
    \monoUp{cat}{feline}{animal}{living thing}
  \end{center}
\end{frame}

\begin{frame}[noframenumbering]{\title}
  \header
  \blurb
  \begin{center}
    \monoUp{feline}{animal}{living thing}{thing}
  \end{center}
\end{frame}

\begin{frame}[noframenumbering]{\title}
  \header
  \blurb
  \begin{center}
    \monoDown{feline}{animal}{living thing}{thing}
  \end{center}
\end{frame}

\begin{frame}[noframenumbering]{\title}
  \header
  \blurb
  \begin{center}
    \monoDown{cat}{feline}{animal}{living thing}
  \end{center}
\end{frame}

\begin{frame}[noframenumbering]{\title}
  \header
  \blurb
  \begin{center}
    \monoDown{house cat}{cat}{feline}{animal}
  \end{center}
\end{frame}

%%%%%%%%%%%%%%%%%%% 
% NATLOG ANIMATION
%%%%%%%%%%%%%%%%%%%
\input example.tex

%%%%%%%%%%%%%%%%%%% 
% BEYOND SYLLOGISMS
%%%%%%%%%%%%%%%%%%%
\input extensions.tex

%%%%%%%%%%%%%%%%%%% 
% ADVANTAGES OF NATLOG
%%%%%%%%%%%%%%%%%%%
\begin{frame}{Properties of Natural Logic}
\begin{itemize}
  \item[\green{\checkmark}] Computationally fast during inference.
  \begin{itemize}
    \item ``Semantic'' parse of query is just syntactic parse.
    \item Inference is lexical mutations / insertions / deletions.
  \end{itemize}
  \vspace{0.5cm}
  \pause

  \item[\green{\checkmark}] Computationally fast during pre-processing.
  \begin{itemize}
    \item Plain text!$^*$
    \pause
    \item[] ($^*$Generated from Ollie extractions.)
  \end{itemize}
  \vspace{0.5cm}
  \pause

  \item[\green{\checkmark}] Still captures common inferences.
  \begin{itemize}
    \item We make these types of inferences regularly and instantly.
    \pause
    \item We expect \textit{readers} to make these inferences instantly.
  \end{itemize}
\end{itemize}
\end{frame}
